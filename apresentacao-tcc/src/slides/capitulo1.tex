\section{Introdu\c{c}\~ao}
	\begin{frame}
	\frametitle{Introdu\c{c}\~ao}
		\begin{itemize}
			\item Sistema de Computa\c{c}\~ao Paralela: conjunto de processadores interconectados conforme alguma topologia para permitir controle de suas atividades e troca de dados \cite{CACERES2001};
			\item Processamento Paralelo: \'e uma forma eficiente de processar informa\c{c}\~ao a qual enfatiza a explora\c{c}\~ao de eventos concorrentes na computa\c{c}\~ao do processo \cite{DIVERIO2002}.
		\end{itemize}
	\end{frame}

	\begin{frame}
	\frametitle{Introdu\c{c}\~ao}
		\begin{itemize}
			\item Utilizado onde grande demanda de poder computacional \'e exigida:
				\begin{itemize}
					\item Previs\~oes metereol\'ogicas;
					\item Simula\c{c}\~oes de engenharia;
					\item Resolu\c{c}\~ao de Sistemas Lineares de grande porte.
				\end{itemize}
			\item M\'aquinas paralelas:
			\begin{itemize}
				\item Alto custo; 
			 	\item Exemplos: Cray-1, Cray X-MP;
			\end{itemize}

			\item Clusters Beowulf (1994): equipamentos de prateleira e {\it software} livre.
		\end{itemize}
	\end{frame}

	\begin{frame}
	\frametitle{Introdu\c{c}\~ao}
		\begin{itemize}
			\item Objetivos deste trabalho:
				\begin{itemize}
					\item Realizar estudo sobre arquiteturas paralelas;
					\item Apresentar modelos para a constru\c{c}\~ao de algoritmos paralelos;
					\item Construir e utilizar ambientes paralelos;
					\item Apresentar a paraleliza\c{c}\~ao de um algoritmo de treinamento de uma Rede Neural Artificial.
				\end{itemize}
		\end{itemize}
	\end{frame}

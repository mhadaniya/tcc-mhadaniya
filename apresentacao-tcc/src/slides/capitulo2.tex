\section{Recupera\c{c}\~ao de Informa\c{c}\~ao}

	\begin{frame}
		\frametitle{\huge Recupera\c{c}\~ao de Informa\c{c}\~ao}
		\large
		\begin{block}{\Large O que \'e?}
			Recupera\c{c}\~ao de Informa\c{c}\~ao (RI) \'e a tarefa de encontrar documentos relevantes a partir de um \textit{corpus} ou conjunto de textos em resposta a uma necessidade de informa\c{c}\~ao de um usu\'ario.% \cite{Smeaton1997}.		
		\end{block}
	\end{frame}
	

	\begin{frame}
	\Large
		\frametitle{\huge Recupera\c{c}\~ao de Informa\c{c}\~ao}
		\framesubtitle{\Large Cl\'assica}
			\begin{itemize}
				\item \textbf{Entrada}: Cole\c{c}\~ao de documentos + Query do usu\'ario
				\item Objetivo: Recuperar documentos ou textos com informa\c{c}\~ao relevantes ao usu\'ario
	
				\item Dois Aspectos:
					\begin{itemize}
					 	\item Processamento da cole��o de documentos;
						\item Processamento de consultas (pesquisa);
					\end{itemize} 
			\end{itemize}
	\end{frame}

	\begin{frame}
		\frametitle{\huge Recupera\c{c}\~ao de Informa\c{c}\~ao}
		\framesubtitle{\Large Cl\'assica}
		\LARGE
			\begin{itemize}
			 	\item Modelo L�gico:
					\begin{itemize}
					 	\item AND, OR, NOT.
					\end{itemize}
			 	\item Modelo L�gico:
					\begin{itemize}
					 	\item + grau de compara��o;
					 	\item fun��o de ordena��o;
					\end{itemize}
			 	\item Modelo L�gico:
					\begin{itemize}
					 	\item Documentos e query representada por vetor de termos;
					\end{itemize}
			 	\item Modelo PLN:
					\begin{itemize}
					 	\item Estrutura e significado ligados;
					\end{itemize}
			\end{itemize}
	\end{frame}
	
	\begin{frame}	
		\frametitle{\huge Recupera\c{c}\~ao de Informa\c{c}\~ao}
		\framesubtitle{\Large Na Web}
		\Large
			\begin{itemize}
				\item Entrada: Web + Query do usu\'ario
				\item Objetivo: Recuperar p\'aginas de alta qualidade com conte\'udos relevantes para o usu\'ario
					\begin{itemize}
					 	\item Est\'atico (texto, \'audio, v\'ideo,\ldots)
						\item Din\^amico
					\end{itemize}
				\item Dois Aspectos:
					\begin{itemize}
					 	\item Processamento e representar a cole\c{c}\~ao de documentos;
							\begin{itemize}
							 	\item Agrupando as p\'aginas est\'aticas
								\item ``Aprender'' sobre p\'aginas din\^amicas
							\end{itemize}
						\item Processamento de consultas (pesquisa);
					\end{itemize}
			\end{itemize}
	\end{frame}
		
	\begin{frame}	
		\frametitle{\huge Recupera\c{c}\~ao de Informa\c{c}\~ao}
		\framesubtitle{\Large Caracter\'isticas da Web}
		\Large
			\begin{itemize}
				\item Tamanho da Internet
				\item Duplica��o - 30\% do conte\'udo \'e c\'opia de algum conte\'udo existente	
				\item Heterogenidade:
					\begin{itemize}
					 	\item Tipos de documentos - texto, figuras, scripts,\ldots;
						\item Qualidade - Desde textos em blogs at\'e artigos cient\'ificos
						\item Idiomas - 100+
					\end{itemize}
				\item M\'ultiplos usu\'arios - cada um utiliza a Internet de uma maneira
				\item Alta Linkagem (High Linkage) - Cada p\'agina cont\'em aproximadamente 8 links para outras p\'aginas
			\end{itemize}
	\end{frame}
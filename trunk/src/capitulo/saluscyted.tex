\chapter{Caso de Uso - Projeto Salus Cyted}\label{Capitulo 04}

	O projeto Salus Cyted nasceu com o objetivo de estimular a participa��o coletiva e estruturar uma s�lida coopera��o entre os grupos participantes, e para tanto, as atividades foram centralizadas em um ambiente espec�fico: um Portal Web com acesso adapt�vel aos diferentes usu�rios em fun��o de seu n�vel cultural e peculiaridades regionais.
	
\section{Cria��o do Parser}
	
	\textbf{Parser} - � um componente de um programa que analisa a estrutura l�xica de uma entrada de acordo com regras pr�-definidas. E para a cria��o do 
	
	A ferramenta para a implementa��o do NameParser foi o JavaCC(Java Compiler Compiler[tm]). � um gerador de parser inicialmente desenvolvido pela Sun. Utiliza uma sintaxe pr�xima do Java, e trabalha \textit{top-down}.
	 
	Um Parser � um programa de computador (ou apenas um componente de um programa) que serve para analisar a estrutura gramatical de uma entrada, manipulando os tokens, que s�o segmentos de texto ou s�mbolos que podem ser manipulados. Em XML, o parser pode ser um leitor que ajuda na convers�o do arquivo para manipula��o dos dados contidos no mesmo.
	
	A cria��o do Parser para a extra��o de nomes em paginas, foi desenvolvida em v�rias etapas. No primeiro momento, foi criado a Express�o Regular para a extra��o de nomes. 
		